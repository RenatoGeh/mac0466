\documentclass{amsart}

\usepackage[brazilian]{babel}
\usepackage[utf8]{inputenc}
\usepackage{graphicx}
\usepackage{mathtools}
\usepackage{amsthm}
\usepackage{thmtools,thm-restate}
\usepackage{amsfonts}
\usepackage{hyperref}
\usepackage[backend=biber,url=true,doi=true,eprint=false,style=alphabetic]{biblatex}
\usepackage{enumitem,multicol}
\usepackage[justification=centering,singlelinecheck=false]{caption}
\usepackage{indentfirst}
\usepackage{algorithm}
\usepackage{algpseudocode}
\usepackage{listings}
\usepackage[x11names, rgb]{xcolor}
\usepackage{tikz}
\usepackage{hyperref}
\usepackage{subcaption}
\usepackage{booktabs}
\usepackage{linegoal}
\usepackage{csquotes}
\usetikzlibrary{snakes,arrows,shapes}

\addbibresource{references.bib}

\makeatletter
\def\subsection{\@startsection{subsection}{3}%
  \z@{.5\linespacing\@plus.7\linespacing}{.1\linespacing}%
  {\normalfont}}
\makeatother

\makeatletter
\patchcmd{\@setauthors}{\MakeUppercase}{}{}{}
\makeatother

\DeclareMathOperator*{\argmin}{arg\,min}
\DeclareMathOperator*{\argmax}{arg\,max}
\DeclareMathOperator*{\Val}{\text{Val}}
\DeclareMathOperator*{\Ch}{\text{Ch}}
\DeclareMathOperator*{\Pa}{\text{Pa}}
\DeclareMathOperator*{\Sc}{\text{Sc}}
\newcommand{\ov}{\overline}
\newcommand{\region}{\mathcal}

\newcommand\defeq{\mathrel{\overset{\makebox[0pt]{\mbox{\normalfont\tiny\sffamily def}}}{=}}}

\newcommand{\algorithmautorefname}{Algorithm}
\algrenewcommand\algorithmicrequire{\textbf{Input}}
\algrenewcommand\algorithmicensure{\textbf{Output}}
\algnewcommand{\LineComment}[1]{\State\,\(\triangleright\) #1}

\captionsetup[table]{labelsep=space}

\theoremstyle{plain}

\newcounter{dummy-def}\numberwithin{dummy-def}{section}
\newtheorem{definition}[dummy-def]{Definition}
\newcounter{dummy-thm}\numberwithin{dummy-thm}{section}
\newtheorem{theorem}[dummy-thm]{Theorem}
\newcounter{dummy-prop}\numberwithin{dummy-prop}{section}
\newtheorem{proposition}[dummy-prop]{Proposition}
\newcounter{dummy-corollary}\numberwithin{dummy-corollary}{section}
\newtheorem{corollary}[dummy-corollary]{Corollary}
\newcounter{dummy-lemma}\numberwithin{dummy-lemma}{section}
\newtheorem{lemma}[dummy-lemma]{Lemma}
\newcounter{dummy-eg}\numberwithin{dummy-eg}{section}
\newtheorem{example}[dummy-eg]{Example}

\theoremstyle{definition}
\newtheorem{exercise}{Exercício}
\newtheorem{answer}{Resposta}

\numberwithin{equation}{section}

\newcommand{\set}[1]{\mathbf{#1}}
\newcommand{\pr}{\mathbb{P}}
\newcommand{\eps}{\varepsilon}
\newcommand{\false}{\text{F}}
\newcommand{\true}{\text{T}}
\newcommand{\expected}{\mathbb{E}}
\renewcommand{\implies}{\Rightarrow}

\newcommand{\bigo}{\mathcal{O}}

\setlength{\parskip}{1em}

\lstset{frameround=fttt,
	numbers=left,
	breaklines=true,
	keywordstyle=\bfseries,
	basicstyle=\ttfamily,
}

\newcommand{\code}[1]{\lstinline[mathescape=true]{#1}}
\newcommand{\mcode}[1]{\lstinline[mathescape]!#1!}
\newcommand{\dset}[1]{\mathcal{#1}}

\title{%
  \noindent\rule{13cm}{1.0pt}\\
  \vspace{0.2cm}
  MAC0466\\
  Teoria dos Jogos Algorítmica\\
  Prova 1
  \noindent\rule{13cm}{0.8pt}
}
\xdef\shorttitle{Prova 1 --- MAC0466}
\author[]{\normalsize\textbf{Renato Lui Geh}\\\small NUSP\@: 8536030}

\begin{document}

\maketitle

\setcounter{exercise}{1}
\setcounter{answer}{1}
\begin{exercise}
  No Jogo de Balanceamento de Carga, temos $n$ tarefas, cada uma com um peso $w_i$, e $m$ máquinas,
  cada uma com uma velocidade $s_i$. Cada jogador escolhe a máquina onde vai sua tarefa. Mostre que
  existe no máximo um equilíbrio de Nash totalmente misto para toda instância do Jogo de
  Balanceamento de Carga. \textit{Dica:} Descreva as condições nas probabilidades $p_i^j$ impostas
  por um equilíbrio de Nash totalmente misto na forma de um sistema de equações lineares e mostre
  que tal sistema tem uma única solução. Deduza daí que há no máximo um equilíbrio destes.
\end{exercise}
\begin{answer}~\\

  $n$ tarefas, $m$ máquinas

  $w_i$: peso da tarefa $i$

  $s_i$: velocidade da máquina $i$\\

  Existe um equilíbrio $S^\star$ pelo Teorema de Nash, já que o jogo é finito. Queremos provar que
  $S^\star$ é único.

  A carga de uma máquina é dada por
  \begin{equation*}
    l_j = \sum_{i=1}^n \frac{w_i x_i^j}{s_j}
  \end{equation*}
  onde $x_i^j$ é uma variável aleatória e igual a 1 se $i$ escolhe a máquina $j$ e $x_i^j=0$ caso
  contrário.

  O custo esperado é
  \begin{equation*}
    c_i^j = \frac{w_i + \sum_{\substack{k\neq i\\k=1}}^n w_k p_k^j}{s_j}
  \end{equation*}
  de $i$ escolher $j$.

  Tirando a esperança de $l_j$:

  \begin{align*}
    \expected[l_j] = \expected\left[\sum_{i=1}^n \frac{w_i x_i^j}{s_j}\right] = \sum_{i=1}^n
      \expected\left[\frac{w_i x_i^j}{s_j}\right] = \sum_i\expected\left[\frac{w_i}{s_j}\right]
      \expected[x_i^j]=\sum_i\expected\left[\frac{w_i}{s_j}\right] p_i^j=\sum_{i=1}^n
      \frac{w_i}{s_j}p_i^j
  \end{align*}

  O custo $c_i^j$ então passa a ser:

  \begin{align*}
    c_i^j = \frac{w_i}{s_j} + \sum_{\substack{k\neq i\\k=1}}\frac{w_k}{s_j} p_k^j=\frac{w_i}{s_j}+
      \expected[l_j]-\frac{w_i}{s_j}p_i^j=\frac{w_i}{s_j}(1-p_i^j)+\expected[l_j]\implies
  \end{align*}
  \begin{equation*}
    \implies (c_i^j-\expected[l_j])\frac{s_j}{w_i}=1-p_i^j\implies p_i^j=1-c_i^j\frac{s_j}{w_i}+
      \expected[l_j]\frac{s_j}{w_i}
  \end{equation*}

  Portanto cada $p_i^j$ é condicionado por:

  \begin{equation*}
    p_i^j=1-c_i^j\frac{s_j}{w_i}+\sum_{k=1}^n\frac{w_k}{s_j}p_k^j\frac{s_j}{w_i}=1-c_i^j
      \frac{s_j}{w_i}+\sum_{k=1}^n\frac{w_k}{w_i}p_k^j
  \end{equation*}
  \begin{equation*}
    p_i^j=1-c_i^j\frac{s_j}{w_i}+p_i^j+\sum_{\substack{k\neq i\\k=1}}^n\frac{w_k}{w_i}p_k^j\implies
      -1+c_i^j \frac{s_j}{w_i}=\sum_{\substack{k\neq i\\k=1}}^n p_k^j
  \end{equation*}

  Resultando no seguinte sistema:

  \begin{equation*}
    \begin{cases}
      \sum_{\substack{k\neq i\\k=1}}^n p_k^j = c_i^j\frac{s_j}{w_i}-1 & \quad 1\leq
      i\leq n, 1\leq j\leq m \quad (\star)\\
      \sum_{j=1}^m p_i^j = 1, & \quad 1\leq i\leq n \quad (\star\star)\\
    \end{cases}
  \end{equation*}

  Em $(\star)$ temos uma equação de $n-1$ variáveis se fixarmos $i$ e $j$. No total, temos $nm$
  equações de $n-1$ variáveis. Em $(\star\star)$ temos uma equação de $m$ variávels fixando $i$.
  Portanto temos no total um sistema de $nm$ equações e $nm$ variáveis. Como a matriz formada pelas
  equações é não-singular (já que $p_i^j>0$), então existe apenas uma solução para o sistema, e
  portanto existe apenas um único equilibrio.

  \hfill\qed%
\end{answer}
%--------------------------------------------------------------------------------------------------

\printbibliography[]

\end{document}
